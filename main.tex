\documentclass[12pt]{article}
\usepackage{geometry}
\geometry{margin=1in}
\usepackage{titlesec}
\usepackage{setspace}
\usepackage{parskip}
\usepackage{lmodern}
\usepackage{sectsty}

\title{Department of Data Science\\[0.5em]
Handbook of Policies \& Procedures\\[0.5em]
William \& Mary\\[0.5em]School of Computing, Data Sciences \& Physics}
\date{\today}
\author{Faculty of the Department of Data Science}

\begin{document}

\maketitle

The \textit{Department of Data Science Handbook of Policies \& Procedures} provides information about departmental operations, serving as a supplement to the university-wide \textit{Faculty Handbook} and relevant policy documents affirmed by the School of Computing Data Sciences \& Physics (CDSP).  

\section*{Faculty Members}

The Department of Data Science recognizes three kinds of faculty members: tenure-track, teaching, and research.

\begin{itemize}
    \item \textbf{Tenure-track faculty} are appointees who are tenured or eligible for tenure.
    \item \textbf{Teaching faculty} are appointees with non-tenure eligible appointments, whether part-time, full-time fixed-term, or full-time renewable, as defined in the CDSP Teaching Faculty Policies.
    \item \textbf{Research faculty} are appointees with full-time, non-tenure eligible appointments in the categories specified in the CDSP Research Faculty Policies.
\end{itemize}

Faculty members in the tenure-track category shall be invited to attend and vote on general matters at Department of Data Science faculty meetings, including remote votes. Teaching and Research faculty shall be extended these privileges only if they hold the rank of Assistant, Associate, or Professor. Faculty members also hold specific voting and governance responsibilities, as further detailed in this document. 

\section*{Committees}

The Chair of the Department of Data Science shall, each year, appoint tenure-track, teaching, and research faculty members to Departmental committees. Committees shall meet, and, when appropriate, vote and affirm matters within their purview by majority vote. Votes may be held remotely. Standing committees shall include:

\begin{itemize}
    \item Undergraduate Program Committee
    \item Graduate Program Committee
    \item Personnel Committee
\end{itemize}

The Chair may create ad-hoc committees as may be necessary.

\section*{Undergraduate Program Committee}

The Undergraduate Program Committee is composed of the Director of Undergraduate Programs (serving as Chair) and at least two additional members appointed by the Department Chair. All committee members must hold Tenure-track or Teaching faculty appointments.

This committee is responsible for overseeing the design, implementation, and evaluation of the department’s undergraduate programs. Its duties - subject to ratification by Department, School or University-wide committees as may be appropriate, include:

\begin{itemize}
    \item Recommending departmental policies and academic requirements for undergraduate programs.
    \item Reviewing and recommending changes to existing undergraduate curricula, including the addition or removal of courses.
    \item Proposing new academic programs or modifications to existing ones, including proposals for new courses or the removal of existing ones.
    \item Reviewing and approving course substitutions, transfer credits, independent studies, and other academic exceptions for individual undergraduate students.
\end{itemize}

\section*{Graduate Program Committee}

The Graduate Program Committee is composed of the Director of Graduate Programs (serving as Chair) and at least two additional members appointed by the Department Chair. All committee members must be Tenure-track or Research faculty.

This committee is responsible for the strategic direction, oversight, and administration of the department’s graduate programs. Its responsibilities include:

\begin{itemize}
    \item Developing and approving departmental policies for graduate programs.
    \item Reviewing and recommending curriculum changes, including proposals for new courses or the removal of existing ones.
    \item Proposing new graduate programs or revisions to current programs of study.
    \item Reviewing and giving Departmental approval for course substitutions, transfer credits, independent studies, and other academic exceptions for individual graduate students.
    \item Assigning and approving advisors.
    \item Conducting annual reviews of Ph.D. student progress.
    \item Coordinating the recruitment and admissions process for incoming Ph.D. student cohorts.
\end{itemize}


\section*{Personnel Committee}

The Personnel Committee of the Department of Data Science consists of four members appointed from the Department faculty. At least three of these members must be tenure-track. 

The responsibilities of the Personnel Committee include:

\begin{itemize}
    \item Developing and recommending departmental policies related to faculty appointment, retention, promotion, tenure, post-tenure review, and joint appointments.  Any policies related to tenure-track faculty must be affirmed by a majority of tenure-track faculty members in the department.
    \item Advising the Department Chair on the assignment of annual merit scores. Evaluations for tenure-track faculty members must be conducted by tenure-track faculty members.
    \item Coordinating and ensuring the completion of annual written teaching observations for all faculty members actively teaching courses.
    \item Recommend membership for Ad Hoc Retention, Promotion, and Tenure committees, pursuant to the policies outlined in this document.
\end{itemize}


\section*{Procedures for Retention, Promotion, Tenure, and Evaluation}

\subsection*{I. Preface}
This section outlines the procedures for evaluating department members with respect to retention, promotion, tenure (RPT), and post-appointment evaluation. All evaluations are conducted with reference to the standards of the William \& Mary Faculty Handbook, CDSP bylaws, and RPT guidelines for CDSP.  Throughout this section, the term \textit{of-rank} shall indicate faculty members who are of an equal or greater rank than the member being evaluated.

\subsection*{II. Role of Committees in Evaluations}

The Personnel Committee is responsible for advising the Chair on the assignment of annual merit evaluation scores. Decisions related to retention (including mid-probationary reviews and post-tenure reviews), promotion, and tenure are conducted by ad hoc RPT Committees.  Membership in these Ad Hoc RPT committees are recommended to the Chair by the Personnel Committee, pursuant to the policies outlined in this document. For tenure-track faculty undergoing interim reviews, evaluation for promotion to Associate Professor, promotion to Professor, or post-tenure review, the committee will consist of at least three of-rank tenured faculty members from the department, selected by majority vote of the tenure-track members of the Personnel Committee. When necessary, qualified faculty from related departments may be invited to ensure disciplinary breadth or meet minimum committee size.  The makeup of the teaching faculty and research faculty ad hoc RPT committees are specified in the relevant sections of this document below.
    
Faculty under review may not participate in their own evaluation.

\subsection*{III. Criteria for Evaluation}

Faculty members are primarily evaluated based on their performance in the following areas:

\begin{itemize}
    \item \textbf{Tenure-track:} Research, teaching, and service/governance.
    \item \textbf{Teaching:} Teaching effectiveness and service/governance.
    \item \textbf{Research:} Research output and impact, grant activity, and professional service.
\end{itemize}


\subsection*{IV. Evidence and Evaluation Process}

\subsubsection*{Annual Evaluations}

Faculty members in the Department of Data Science are evaluated annually based on their contributions during the previous calendar year (January 1 to December 31).\footnote{The Department will not conduct annual evaluations for faculty who report solely to the Dean, Provost, President, or Board of Visitors; such evaluations will be the responsibility of the appropriate supervising office.} Faculty will submit an annual merit evaluation form describing research, teaching, and service activities that is evaluated alongside supporting documentation, including teaching evaluations and evidence of impact. Submitted materials are reviewed by the Department Personnel Committee, which advises the Chair as to scores and provides narrative feedback.  The Chair provides the final scores to the Office of the Dean.

\paragraph{Scoring Models}
Annual evaluations use distinct models depending on faculty type:

\begin{itemize}
    \item \textbf{Tenure-track faculty:} Evaluated on a 7-5-3 scale\footnote{This scale may be modified for individual tenured faculty members. Any such modification must be initiated by the faculty member and approved in writing by the Department Chair and Dean.  Individuals in a tenure track position who have not yet been awarded tenure are not eligible to modify this scale.}:
    \begin{itemize}
        \item 7 points for research (e.g., peer-reviewed publications, grants).
        \item 5 points for teaching (e.g., course evaluations, advising).
        \item 3 points for service (e.g., committee work, university service).
    \end{itemize}
    \item \textbf{Teaching faculty:} Full-time teaching faculty will be evaluated on a 6-3 scale:
    \begin{itemize}
        \item 6 points for teaching.
        \item 3 points for service.
    \end{itemize}
    Part-time teaching faculty will receive a rating of ``Exceeds Expectations", ``Meets Expectations", or ``Does Not Meet Expectations" on the basis of their performance.
    \item \textbf{Research faculty:} Evaluated out of 6 points for research. If the faculty member’s contract includes teaching responsibilities, teaching will also be evaluated on a 6-point scale. If service is included in the contract, it will be evaluated on a 3-point scale. Postdoctoral fellows and part-time research faculty are not evaluated using this numeric system; instead, they receive a rating of ``Exceeds Expectations,” ``Meets Expectations,” or ``Does Not Meet Expectations” based on performance. 
    
\end{itemize}

\paragraph{Teaching Evaluation}
Teaching performance is evaluated holistically. While numerical course evaluations are a component, scores are considered in context. The committee will account for factors such as course level and enrollment size when interpreting evaluation data. Faculty are encouraged to include evidence of teaching impact beyond evaluations, including mentoring, curriculum development, or teaching awards.  Metrics such as graduate students advised or undergraduate research theses should be included as an element of teaching.

\paragraph{Research Evaluation}
For tenure-track and research faculty, research evaluation includes published and accepted peer-reviewed work, funded projects, invited and contributed presentations, other forms of dissemination and related activities.  

\paragraph{Service Evaluation}
Service is assessed based on contributions to departmental governance, service to CDSP, university service, and disciplinary engagement (e.g., editorial boards, proposal reviews, conference organization). 

\paragraph{Evaluation Outcome}
Each faculty member will receive a merit score and brief written feedback from the Chair. The distribution of all scores in the department will also be summarized (i.e., by mean) and provided to each faculty member. If the Chair substantively changes the scores or narrative recommended by the Personnel Committee, such changes must be noted in the summary provided to each faculty member. For tenure-track faculty, an unsatisfactory annual merit review score is defined in terms of the three numbers assigned to a department member for research, teaching, and governance in the merit review process described in this section. An annual merit evaluation will be deemed overall unsatisfactory if the sum of the department member's three numbers is seven or less on a scale of zero to fifteen.  For part-time faculty and postdoctoral fellows, annual merit evaluation will be deemed unsatisfactory if they receive a rating of ``Does Not Meet Expectations."

\subsubsection*{Retention, Tenure, or Promotion for Tenure-track Faculty}

For any retention, tenure, or promotion case, the faculty member under review must submit at minimum the following materials:

\begin{itemize}
    \item A current curriculum vitae.
    \item A narrative self-assessment addressing accomplishments in teaching, research, and service, as appropriate to the faculty member.
    \item Copies of published and accepted peer-reviewed scholarly work.
    \item Copies and summaries of awarded grants, contracts, or other external funding.
    \item Student course evaluations and at least one supplemental piece of evidence of teaching effectiveness (for example, a course observation).
\end{itemize}

Additional materials may be included at the discretion of the candidate, or as required by the Faculty Handbook or relevant CDSP policies.

\textit{Tenure and Promotion for Tenure-Track Faculty}\\
For tenure and promotion cases involving tenure-track faculty, at least four external review letters shall be obtained.\footnote{With exceptions being allowed in cases where tenure may have previously been awarded by another organization; such exceptions must be approved by the Chair and the Dean, included as a note in the Chair's letter for the case, and may only be awarded if such exceptions conform with all relevant School and University guidance.} The process for selecting external reviewers will proceed as follows:

\begin{enumerate}
    \item The committee will generate a list of a minimum of six potential external reviewers who meet the ``arm’s length'' criteria as defined by the CDSP Procedures on Tenure, Promotion and Interim Review Processes, Faculty Handbook and associated university guidance.
    \item The candidate will prepare a separate list of potential ``arm’s length'' external reviewers.
    \item The candidate may remove up to two names from the committee’s list and may provide additional context about any proposed reviewers.
    \item From the combined pool of eligible names, the committee will select at least four external reviewers from whom to solicit letters; additional reviewers can be solicited in order to ensure a minimum of four letters are received in a timely fashion. The final list will not be disclosed to the candidate.
\end{enumerate}

After the external letters have been received and anonymized, the candidate will be given seven (7) calendar days to submit a written response providing rebuttal or clarification to the external letters. Following this, the committee will evaluate the candidate’s full portfolio - including submitted materials, external letters, and any response from the candidate - and prepare a written assessment and recommendation.

The committee’s report will include:

\begin{itemize}
    \item A summary of evidence and rationale supporting the recommendation.
    \item The outcome of the committee vote, including any dissenting opinions if applicable.
\end{itemize}

The committee report will be submitted to the candidate, who will be provided with seven (7) calendar days to write a reply to the report.  The full dossier, inclusive of the committee report and all candidate responses to both external letters and the committee report, will then be submitted to all of-rank tenure-track faculty members. In a meeting chaired by the department chair, the of-rank faculty shall then hold a vote affirming or declining the recommendation of the Personnel Committee.  The department chair shall participate in this vote, and write a recommendation on behalf of the faculty. Both the committee's and faculty's recommendations will be submitted to the CDSP Committee on Retention, Promotion, and Tenure.

\textit{Interim Review for Tenure-Track Faculty}\\
For tenure-track faculty members undergoing an interim review, the committee will consider the candidate’s submitted portfolio, and prepare an assessment and recommendation.  In a meeting chaired by the department chair, the of-rank faculty shall hold a discussion and vote affirming or declining the recommendation of the committee.  The department chair shall participate in this vote, and write a recommendation on behalf of the faculty.  This recommendation will be submitted to the CDSP Committee on Retention, Promotion and Tenure, and will include:

\begin{itemize}
    \item A summary of evidence and rationale supporting the recommendation.
    \item The outcome of the faculty vote, including any dissenting opinions if applicable.
    \item Constructive feedback to the candidate regarding their progress towards tenure.
\end{itemize}


\subsubsection*{Post-Tenure Review}
Post-tenure review can be conducted at the request of the Dean, or if a tenured faculty member receives an overall unsatisfactory annual merit review in two consecutive years.  When a post-tenure review is required, it will be conducted by a committee of three of-rank tenured faculty members from the Department of Data Science, recommended by the of-rank members of the Personnel Committee. The committee will evaluate the faculty member’s performance in research, teaching, and service, in accordance with departmental and university expectations. 

The faculty member under review will submit the same materials required for a promotion with tenure, with the exception that external letters of evaluation will not be solicited.

The committee will prepare a written report addressed to both the faculty member and the Department Chair. This report will include:

\begin{itemize}
    \item A summary of the evidence reviewed and the rationale for the committee’s recommendation.
    \item The outcome of the committee vote, including any dissenting opinions if applicable.
\end{itemize}

The faculty member will have seven calendar days to submit a written response to the report to the Chair.

Following this, the Department Chair will prepare a separate summary letter evaluating the case, based on the committee’s report and the faculty member’s response. If the Chair disagrees with the committee’s findings, this must be explicitly stated, and the original committee report must be appended to the Chair’s letter. 

The final report will be submitted to the Dean for further action, pursuant to the Faculty Handbook of William \& Mary.

\subsubsection*{Promotion for Teaching Faculty}

Teaching faculty in renewable positions may be considered for promotion in rank in accordance with CDSP policy. Promotion is not automatic and must be initiated by the faculty member in consultation with the Department Chair. Faculty may remain at their current rank indefinitely, provided they continue to meet expectations and are renewed through standard procedures. Promotion is ordinarily considered during the second renewal cycle at a given rank, though in exceptional cases it may be considered at the first renewal. 

Promotion requires a sustained record of excellence in teaching and service. Evaluations of teaching performance will consider multiple sources, including but not limited to: peer observations, student evaluations, curricular development, and supervision of student research or projects. Excellence in teaching is demonstrated through consistent effectiveness in the classroom; clear, well-designed and current course materials; inclusive and evidence-based pedagogy; meaningful mentoring and advising; sustained contributions to curricular innovation and assessment; and documented student learning, engagement, and success over time. Excellence in service is demonstrated through meaningful, diversified engagement in departmental, CDSP, or university service roles. Service contributions are evaluated proportionally to the service load specified by rank and appointment.

Promotion dossiers for Teaching faculty must include the following:

\begin{itemize}
    \item A current curriculum vitae.
    \item A narrative statement (maximum three pages) reflecting on growth, teaching, and service contributions since appointment to the current rank.
    \item Evidence of excellence in teaching, drawing from a minimum of two different modalities of evidence (i.e., course evaluations and peer reports).
\end{itemize}

The dossier will be evaluated by an ad hoc committee, which will prepare a written report and recommendation. Members of the committee must either hold tenure, or be at the rank of Teaching Associate Professor or higher, with membership recommended by the Personnel Committee. The Committee's evaluation will be shared with the candidate, who will have no fewer than seven calendar days to submit a written response before the dossier, committee report, and candidate response (if provided) are forwarded to all Data Science faculty who hold a rank at or higher than the faculty candidate. These faculty will vote to recommend or withhold recommendation for promotion. On the basis of this vote and all material, the Department Chair will then prepare a separate letter of evaluation and send the full dossier to the Dean. The final decision on promotion rests with the Dean.

\subsubsection*{Promotion for Research Faculty}

Research faculty may be considered for promotion in accordance with CDSP policy. Postdoctoral researchers, while ineligible for promotion, may apply for research faculty positions as may be appropriate. Promotion is not automatic and must be initiated by the faculty member in consultation with the Department Chair. Faculty may remain at their current rank as long as they hold a valid appointment, provided they continue to meet expectations and funding is available to support the position.

Promotion may be considered at the time of contract renewal and is contingent on a successful renewal decision. A history of excellent performance is required for promotion. Excellence is demonstrated through a sustained record of achievement in research, including, but not limited to: peer-reviewed publications, acquisition of external grants, conference presentations (invited and contributed), mentorship of students or postdocs, and recognition through honors or awards. Consistent annual merit ratings of “meets expectations” are considered evidence of excellence but are not sufficient on their own.

Promotion dossiers must include the following:

\begin{itemize}
    \item A current curriculum vitae.
    \item A narrative statement (maximum three pages) describing the faculty member’s growth and contributions in research since their appointment to the current rank.
\end{itemize}

The dossier will be reviewed by the an Ad Hoc Committee consisting of tenured or tenure-eligible members of the Department, which will prepare a written report and recommendation. The makeup of this Ad Hoc Committee will be recommended to the Chair by the Personnel Committee. The report will be shared with the candidate, who will have no fewer than seven calendar days to respond to the committee's report, and any such response will be forwarded alongside the full dossier for a vote by all tenured and tenure-eligible members of the Department of Data Science affirming or declining to affirm the committee recommendation. The Department Chair will then prepare a letter of representing the recommendation of the Faculty. The candidate will also have no fewer than seven calendar days to respond to the Chair’s letter before the dossier is submitted to the Dean. The final decision on promotion rests with the Dean.


\subsection*{VII. Recordkeeping and Amendments}

The Department will maintain individual personnel files and a general record of all committee actions related to evaluation and advancement. Any faculty member may propose amendments to these procedures. Amendments require a two-thirds affirmative vote of the eligible faculty, approval from the Dean, and may require further approval by other relevant school and university bodies to take effect. Any amendments impacting policies related to tenure-track faculty must be affirmed by a two-thirds majority of tenure-track faculty.




\end{document}
